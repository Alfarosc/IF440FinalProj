\documentclass[conference]{IEEEtran}
\IEEEoverridecommandlockouts
% The preceding line is only needed to identify funding in the first footnote. If that is unneeded, please comment it out.
\usepackage{cite}
\usepackage{amsmath,amssymb,amsfonts}
\usepackage{algorithmic}
\usepackage{graphicx}
\usepackage{textcomp}
\usepackage{xcolor}

\makeatletter
\def\endthebibliography{%
  \def\@noitemerr{\@latex@warning{Empty `thebibliography' environment}}%
  \endlist}
\makeatother

\def\BibTeX
    {
        {\rm B\kern-.05em{\sc i\kern-.025em b}
        \kern-.08em
        T\kern-.1667em\lower.7ex\hbox{E}\kern-.125emX
        }
    }

    
\begin{document}

\title
    {
        Implementasi Teknologi Pengenalan Gambar Untuk Meningkatkan Keamanan Sistem Komputer Dengan Mendeteksi Dan Mengidentifikasi Wajah
    }

\author
    {
            \IEEEauthorblockN{1\textsuperscript{st} Andrew Thomas Agustinus}
            \IEEEauthorblockA{\textit{Fakultas Teknik dan Informatika (of Aff.)} \\
            \textit{Universitas Multimedia Nusantara (of Aff.)}\\
            Tangerang, Indonesia\\
            andrew.thomas@student.umn.ac.id}
        \and
            \IEEEauthorblockN{2\textsuperscript{nd} Arvin Winardi}
            \IEEEauthorblockA{\textit{Fakultas Teknik dan Informatika (of Aff.)} \\
            \textit{Universitas Multimedia Nusantara (of Aff.)}\\
            Tangerang, Indonesia\\
            arvin.winardi@student.umn.ac.id}
        \and
            \IEEEauthorblockN{3\textsuperscript{rd} Axel Ferdinand}
            \IEEEauthorblockA{\textit{Fakultas Teknik dan Informatika (of Aff.)} \\
            \textit{Universitas Multimedia Nusantara (of Aff.)}\\
            Tangerang, Indonesia\\
            axel.ferdinand@student.umn.ac.id}
        \and
            \IEEEauthorblockN{4\textsuperscript{th} Mohammad Alfarizky Ramadhani Oscandar}
            \IEEEauthorblockA{\textit{Fakultas Teknik dan Informatika (of Aff.)} \\
            \textit{Universitas Multimedia Nusantara (of Aff.)}\\
            Tangerang, Indonesia\\
            mohammad.alfarizky@student.umn.ac.id}
    }
% ^^^^^^^^^^^^^^^^^^^^^^^^^^^^^^^^^^^^^^^^^^^^^^^^^^^^^^^^^^^^^^^^^^^^^^^^^^^^^^^^^^^^^^^^^^^^
\maketitle
% ^^^^^^^^^^^^^^^^^^^^^^^^^^^^^^^^^^^^^^^^^^^^^^^^^^^^^^^^^^^^^^^^^^^^^^^^^^^^^^^^^^^^^^^^^^^^
 

% ############################################################################################

\begin{abstract}
    This document is a model and instructions for \LaTeX.
    This and the IEEEtran.cls file define the components of your paper [title, text, heads, etc.]. *CRITICAL: Do Not Use Symbols, Special Characters, Footnotes, 
    or Math in Paper Title or Abstract.
\end{abstract}

\begin{IEEEkeywords}
    component, formatting, style, styling, insert
\end{IEEEkeywords}

% ############################################################################################

\section{Pendahuluan}
\subsection{Latar Belakang}
Di era digital saat ini, sistem komputer menjadi kebutuhan manusia untuk membantu menyelesaikan berbagai pekerjaan penting dalam kehidupan sehari-hari. Mulai dari komunikasi, menyelesaikan masalah, menjalankan bisnis dan mengelola data menjadikannya sangat dibutuhkan di kehidupan manusia. Peningkatan konektivitas jaringan komputer di dunia memberikan dampak yang berpengaruh untuk kehidupan manusia. Dengan demikian, komputer saat ini tidak hanya dapat membantu manusia untuk menyelesaikan tugas-tugas konvensional seperti komunikasi, mengetik, merangkum informasi, dan lainnya. Namun, komputer saat ini dapat membantu manusia dalam mengambil keputusan dan meringkas pekerjaan-pekerjaan konvensional sebelumnya dengan teknologi kecerdasan buatan.

Teknologi kecerdasan buatan tidak hanya memberikan dampak yang positif untuk kehidupan manusia. Namun, dengan pesatnya peningkatan kecerdasan buatan juga dapat menimbulkan masalah keamanan di lingkungan digital. Keterbatasan manusia untuk mengolah informasi dan memiliki bias framing dalam mempersepsikan situasi menjadikannya rentan untuk dieksploitasi oleh program-program jahat. Salah satu kemungkinan celah keamanan yang menjadi masalah yaitu program robot yang menirukan manusia untuk mendapatkan keuntungan dalam memproses tindakan yang seharusnya dilakukan oleh manusia. Dengan demikian, diperlukan adaptasi kecerdasan buatan untuk menghadapi hal tersebut dari segi keamanan sistem komputer.

Tindakan keamanan konvensional seperti verifikasi, kata sandi, dan enkripsi tidak lagi cukup untuk melindungi pengguna. Oleh karena itu, penerapan sistem keamanan dengan kecerdasan buatan berlandaskan genetic algorithm mulai dikembangkan sebagai pendekatan menjanjikan untuk sistem keamanan komputer. Penerapan tersebut dapat dipadukan dengan penerapan teknologi terkini, seperti pengenalan gambar dengan deteksi dan identifikasi objek. Artikel ini mengeksplorasi pemanfaatan teknologi pengenalan citra yang dikombinasikan dengan algoritma genetik untuk mendeteksi dan mengidentifikasi objek, sehingga meningkatkan keamanan sistem komputer secara keseluruhan.

\subsection{Tujuan Penelitian}
Penelitian ini bertujuan untuk mengevaluasi efisiensi dan keandalan teknologi pengenalan gambar dalam mendeteksi dan mengidentifikasi objek dalam konteks keamanan sistem komputer. Melibatkan pengujian dan analisis performa teknologi tersebut, dengan memperhatikan tingkat akurasi, kecepatan deteksi, dan kemampuan adaptasi terhadap berbagai objek.

\subsection{Kajian Teori}
Face recognition adalah cara untuk mengenali wajah manusia melalui bantuan teknologi AI. Sebuah sistem yang digunakan dalam face recognition menggunakan biometrik untuk memetakan fitur wajah dari sebuah foto atau video. Sistem ini membandingkan informasi tersebut dengan database wajah yang dikenal untuk mencari kecocokan. Kecocokan ini didapatkan dengan menggunakan algoritma yang dibentuk melalui GA.

\begin{itemize}
    \item Framework untuk Face Recognition
    \item Pengaplikasian dari Face Recognition
    \begin{itemize}
        \item Keamanan publik
        \item Penegakan hukum
        \item Verifikasi kartu kredit
        \item Kontrol Akses
        \item Human-Computer intelligent interaction
        \item Perpustakaan digital
    \end{itemize}
    \item Tantangan di bidang Face Recognition
    \begin{itemize}
        \item Tantangan 1
        \item Tantangan 2
        \item Tantangan 3
    \end{itemize}
    \item Kelebihan dari penggunaan Face Recognition
\end{itemize}

\subsection{Rumusan Masalah}

% ############################################################################################

\section{Metode Penelitian}
\subsection{Encoding}
Genetic Algorithm memerlukan inisialisasi populasi data untuk menghasilkan pengenalan yang optimal. Dalam membentuk populasi, diperlukan cara untuk merumuskan chromosome yang akan merepresentasikan baris-baris array. Setiap kolom pada chromosome disebut sebagai individu atau gen-gen dari susunan bilangan biner yang merepresentasikan isi dari data yang dibutuhkan. 

Setiap biner yang tercatat pada masing-masing gen akan merepresentasikan parameter untuk fitur-fitur yang diterapkan. Untuk menerapkan fungsi pengenalan objek, umumnya setiap biner memiliki parameter untuk mengelompokkan pola dalam pixel sebuah gambar objek. Dalam penerapan pengenalan wajah, dipertimbangkan ukuran gambar dalam file database 64x64 pixel. Dengan demikian, dapat diberikan setiap solusi yang dihasilkan oleh chromosome akan mewakili posisi (x, y) dari kelompok-kelompok pixel posisi gambar. Dari gambaran tersebut, dapat diberikan panjang kromosom 8 bit, dengan 4 bit untuk koordinat x, dan 4 bit untuk koordinat y.

\subsection{Initial Population}

\subsection{Evaluate Fitness}

\subsection{Parent Selection}

\subsection{Crossover}

\subsection{Mutation}

% ############################################################################################

\section{Hasil}
\cite{Nayak2021}

\section{Pembahasan}
% Lorem ipsum dolor sit amet\cite{Nayak2021}
% \begin{equation}
% a+b=\gamma\label{eq}
% \end{equation}

% \subsection{Figures and Tables}
% \paragraph{Positioning Figures and Tables} 
% Lihat gambar : ``Fig.~\ref{fig2}''
% Place figures and tables at the top and 
% bottom of columns. Avoid placing them in the middle of columns. Large 
% figures and tables may span across both columns. Figure captions should be 
% below the figures; table heads should appear above the tables. Insert 
% figures and tables after they are cited in the text. Use the abbreviation 
% ``Fig.~\ref{fig}'', even at the beginning of a sentence.

% \begin{table}[htbp]
% \caption{Table Type Styles}
% \begin{center}
% \begin{tabular}{|c|c|c|c|}
% \hline
% \textbf{Table}&\multicolumn{3}{|c|}{\textbf{Table Column Head}} \\
% \cline{2-4} 
% \textbf{Head} & \textbf{\textit{Table column subhead}}& \textbf{\textit{Subhead}}& \textbf{\textit{Subhead}} \\
% \hline
% copy& More table copy$^{\mathrm{a}}$& &  \\
% \hline
% \multicolumn{4}{l}{$^{\mathrm{a}}$Sample of a Table footnote.}
% \end{tabular}
% \label{tab1}
% \end{center}
% \end{table}

% \begin{figure}[htbp]
% \centerline{\includegraphics{figures/fig1.png}}
% \caption{Example of a figure caption.}
% \label{fig}
% \end{figure}

% \begin{figure}
%     \centering
%     % \includegraphics[width=0.48\textwidth]{fig_2.png}
%     \caption{Breeding Phase}
%     \label{fig2}
% \end{figure}

% ############################################################################################

% \section*{References}

% Please number citations consecutively within brackets \cite{b1}. The 
% sentence punctuation follows the bracket \cite{b2}. Refer simply to the reference 
% number, as in \cite{b3}---do not use ``Ref. \cite{b3}'' or ``reference \cite{b3}'' except at 
% the beginning of a sentence: ``Reference \cite{b3} was the first $\ldots$''

% Number footnotes separately in superscripts. Place the actual footnote at 
% the bottom of the column in which it was cited. Do not put footnotes in the 
% abstract or reference list. Use letters for table footnotes\cite{Nayak2021}. 

% Unless there are six authors or more give all authors' names; do not use 
% ``et al.''. Papers that have not been published, even if they have been 
% submitted for publication, should be cited as ``unpublished'' \cite{b4}. Papers 
% that have been accepted for publication should be cited as ``in press'' \cite{b5}. 
% Capitalize only the first word in a paper title, except for proper nouns and 
% element symbols.

% For papers published in translation journals, please give the English 
% citation first, followed by the original foreign-language citation \cite{b6}.

% \begin{thebibliography}{00}
% \bibitem{b1} G. Eason, B. Noble, and I. N. Sneddon, ``On certain integrals of Lipschitz-Hankel type involving products of Bessel functions,'' Phil. Trans. Roy. Soc. London, vol. A247, pp. 529--551, April 1955.
% \bibitem{b2} J. Clerk Maxwell, A Treatise on Electricity and Magnetism, 3rd ed., vol. 2. Oxford: Clarendon, 1892, pp.68--73.
% \bibitem{b3} I. S. Jacobs and C. P. Bean, ``Fine particles, thin films and exchange anisotropy,'' in Magnetism, vol. III, G. T. Rado and H. Suhl, Eds. New York: Academic, 1963, pp. 271--350.
% \bibitem{b4} K. Elissa, ``Title of paper if known,'' unpublished.
% \bibitem{b5} R. Nicole, ``Title of paper with only first word capitalized,'' J. Name Stand. Abbrev., in press.
% \bibitem{b6} Y. Yorozu, M. Hirano, K. Oka, and Y. Tagawa, ``Electron spectroscopy studies on magneto-optical media and plastic substrate interface,'' IEEE Transl. J. Magn. Japan, vol. 2, pp. 740--741, August 1987 [Digests 9th Annual Conf. Magnetics Japan, p. 301, 1982].
% \bibitem{b7} M. Young, The Technical Writer's Handbook. Mill Valley, CA: University Science, 1989.
% \end{thebibliography}
% \vspace{12pt}
% \color{red}
% IEEE conference templates contain guidance text for composing and formatting conference papers. Please ensure that all template text is removed from your conference paper prior to submission to the conference. Failure to remove the template text from your paper may result in your paper not being published.

% ############################################################################################
\bibliographystyle{IEEEtran}
\bibliography{biblio}

\end{document}
